\documentclass[12pt]{article}
\usepackage[utf8]{inputenc}
\usepackage{amsmath}
\usepackage{amssymb}
\usepackage{amsthm}
\usepackage{fullpage}

% Python Code
\usepackage{listings, textcomp, color, verbatim}
\definecolor{deepgreen}{rgb}{0,0.4,0}

\lstdefinestyle{python}{
		language=python,
		basicstyle=\color{black}\ttfamily\footnotesize,
		stringstyle=\color{deepgreen}\slshape,
		commentstyle=\color{gray}\slshape,
		keywordstyle=\color{red}\bf,
		emphstyle=\color{blue}\bf,
		tabsize=2,
		%%%%%%%%%%%%%%%
		showstringspaces=false,
    emph={access,and,break,class,continue,def,del,elif,else,except,exec,finally,for,from,global,if,import,in,i s,lambda,not,or,pass,print,raise,return,try,while,as},
		upquote=true,
		morecomment=[s]{"""}{"""},
		literate=*
		{:}{{\textcolor{blue}:}}{1}%
		{=}{{\textcolor{blue}=}}{1}%
		{-}{{\textcolor{blue}-}}{1}%
		{+}{{\textcolor{blue}+}}{1}%
		{*}{{\textcolor{blue}*}}{1}%
		{!}{{\textcolor{blue}!}}{1}%
		{(}{{\textcolor{blue}(}}{1}%
		{)}{{\textcolor{blue})}}{1}%
		{[}{{\textcolor{blue}[}}{1}%
		{]}{{\textcolor{blue}]}}{1}%
		{<}{{\textcolor{blue}<}}{1}%
		{>}{{\textcolor{blue}>}}{1},%
		%%%%%%%%%%%%%%%%
		aboveskip=\baselineskip,
		xleftmargin=20pt, xrightmargin=15pt,
		numbers=left, numberstyle=\tiny
}
\lstnewenvironment{python}{\lstset{style=python}}{}
\newcommand{\inputsamplepython}[1]{\lstinputlisting[style=python]{../ClassCodes/#1}}
\newcommand{\inputpython}[1]{\lstinputlisting[style=python]{#1}}


\begin{document}


\begin{center}
    Tetiana Parshakova 
\end{center}
\begin{center}
    \Large CME211 HW4
\end{center}

\section{Truss class}


Truss is a class for computing truss beams forces for a given
	configuration of joints and beams.

By using the equilibrium equations,  we form a linear system and obtain the
	necessary unknowns, i.e.
\begin{align*}
    & \sum_{j \in N(i)} B_j \cos(theta_j) + R_i^1 \mathbf{1}(i \text{ rigid}) = F^1_i,\\
    & \sum_{j \in N(i)} B_j \sin(theta_j) + R_i^2 \mathbf{1}(i \text{ rigid}) = F^2_i.
\end{align*}


\begin{python}
def compute_forces(self)
\end{python}
computes the beam forces and support forces by forming a sparse linear system. Populates \texttt{self.beam\_forces} variable. We use CSR format for the sparse matrix, and use \texttt{A = csr\_matrix((vals, col\_idx, row\_idx), shape=(2*njoints, nbeam+2*nR))} to solve the sparse system from \texttt{scipy}.


\begin{python}
def read_data(self, fjoints, fbeam)
\end{python}
Reads data from joints file and beam file. Returns the output lists of tuples. Joints and beams start from 1 and continue to increase consecutively; and list of neighboring joints
\begin{python}
			joints_list[i] = (xi, yi, Fix, Fiy, Ri!=0)
			beams_list[k] = (i,j)
			connections[i] = [(k,j)]
\end{python}



\begin{python}
def get_angle(self, x1, x2)
\end{python}
Returns the angle of inclination of the beam. Used to make the beam force parallel to the beam.



\begin{python}
def PlotGeometry(self)
\end{python}
Plot joints and beams and stores them into a file.

\end{document}
